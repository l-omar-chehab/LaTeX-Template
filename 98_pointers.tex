% !TEX root = 00_main.tex

\section{Pointers on writing a good paper}

\paragraph{Pointers on the writing style} A fluid writing style is what makes a paper accessible and memorable. It is very much a method. A few general pointers are
%
\begin{itemize}
    
    \item[] \textbf{write short sentences}
    Make them as easy to understand as possible; spare the reader any additional cognitive load. 
    
    \item[] \textbf{being unclear is much more dangerous than being redundant}
    Eliminating redundancy is not always desirable. When used properly, it helps drive a point and remind the reader of something that was said before. 

    \item[] \textbf{do not break the flow}
    Do not use ``citep" citations in the middle of a sentence: it breaks the flow. Either incorporate the citation in the sentence with a ``citet", or place it at the end with a ``citep". \\
    \textit{e.g. (incorrent)} \ldots \\
    \textit{e.g. (correct)} \ldots 
    
\end{itemize}
% 
Here are more specific pointers on language:
%
\begin{itemize}
    
    \item[] \textbf{common formulations} you ``generalize" an equation and ``relax" a constraint. A definition is ``implicit", not an equation. An explanation can be ``self-contained" or ``standalone". It may start with a ``blueprint" and use ``shorthand" notations.

    \item[] \textbf{link words} sometimes, we wish to present an explanation using language that is a bit less dry than the mathematical link words (``suppose", ``therefore", ``it follows"). You may then use ``actually", ``in fact", or ``importantly". 

    \item[] \textbf{use a citation as a plural} We would say
    Citations referring to the authors require a plural, \textit{e.g.} ``\citet{pihlaja2010nce} offer an answer". Numerical citations can be singular, \textit{e.g.} ``[1] offers an answer". 

\end{itemize}
% 



\paragraph{Pointers on technical vocabulary} Technical terms have different meanings for different scientific communities: any choice is fine so long as it is \textit{coherent}. Here are a few examples to keep in mind:
%
\begin{itemize}
    
    \item[] a \textbf{sample}
    means the whole dataset $(x_1, \ldots, x_n)$, in statistics. It means a single data point $x_i$ in signal processing. In machine learning, it is used loosely and could mean one or the other.

    \item[] a \textbf{distribution}
    refers to a ``probability" distribution or normalized density in statistics. In mathematics, it refers to the more general concept of a ``Schwarz" distribution or generalized function.
    
\end{itemize}

